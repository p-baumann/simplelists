
\documentclass{article}

\usepackage[utf8]{inputenc}
\usepackage[T1]{fontenc}
\usepackage{enumitem}  
\usepackage{simplelists} 
\usepackage{xcolor}
\usepackage{listings}
\lstset
{
    language=[LaTeX]TeX,
    breaklines=true,
    morekeywords={lstinline, itemize, enumerate, ul, Ul, li, Ol, ol, Uli, uli, Oli, oli},
    basicstyle=\tt,
    keywordstyle=\color{blue}
}

\title{\textbf{simplelists}: HTML-style list commands in \LaTeX{}}
\author{Peter Baumann}
\date{}

\begin{document}
\maketitle\noindent
\textbf{simplelists} is a package for creating \lstinline!itemize! and \lstinline!enumerate! lists with short (or terse) HMTL-like commands:
\ul
\li Unordered lists are created with \lstinline!\ul!
\li They are \lstinline!itemize! environments
\ul
\li They nest
\li \lstinline!\li! is the command for list items
\li It is just an alias for \lstinline!\item!
\li[*] So it accepts an optional label argument: \lstinline!\li[*]!
\Ul
\li List environments are closed with the first letter in uppercase: \lstinline!\Ul!
\Ul

\ol
\li Ordered lists are created with \lstinline!\ol!
\li They are \lstinline!enumerate! environments
\ol
\li They nest as well
\ul 
\li And unordered lists can appear nested within them 
\Ul
\Ol
\li They are also closed with the first letter in uppercase: \lstinline!\Ol!
\Ol

\section*{Combined environment and item commands}
For convenience, there are combinations of \lstinline!\ul! and \lstinline!\li!:
\uli A list can be opened with an item in one command: \lstinline!\uli!
\li The following items are again introduced with \lstinline!\li!
\Uli Similarly, the last item can close a list environment: \lstinline$Uli$. 

The empty line following \lstinline$Uli$ is mandatory, the environment is actually closed by the end of a paragraph. To close a nested environment the regular (item-free) closing commands (\lstinline!\Ul! and \lstinline!\Ol!) are recommended. They work with the (combined) opening commands taking an item (\lstinline$uli$). 

The combined closing commands \lstinline!\Uli! and \lstinline!\Oli! are a bit sensitive and may not work in all cases\footnote{It is apparently impossible including an \texttt{lstinline} command with an exclamation mark as separator inside a final item. But then \texttt{lstinline} is impossible in footnotes, no matter which separator is used. The combined commands are possible in footnotes (as long as there is an empty line):
\uli Opening item
\Uli Closing item

}:  The combined opening commands seem to be less sensitive.

\section*{Customization with enumitem}
Both environments play nicely with the package \textbf{enumitem}:
\ul[label=\textregistered, leftmargin=*]
\li Optional arguments passed to \lstinline!\ul! and \lstinline!\ol! 
\li are passed to \lstinline!\begin{itemize}! and \lstinline!\begin{enumerate}!, respectively
\Ul
This also works with the combined commands:
\uli[label=\textregistered, leftmargin=*] Optional arguments passed to \lstinline$\uli$ and \lstinline$\oli$
\Uli are passed to their respective environments.

\end{document}